\documentclass[pastel]{hipstercv}
% available options are: darkhipster, lighthipster, pastel, allblack, grey, verylight
\usepackage[utf8]{inputenc}
\usepackage[default]{raleway}
\usepackage[margin=1cm, a4paper]{geometry}


%------------------------------------------------------------------ Variablen

\newlength{\rightcolwidth}
\newlength{\leftcolwidth}
\setlength{\leftcolwidth}{0.3\textwidth}
\setlength{\rightcolwidth}{0.65\textwidth}

%------------------------------------------------------------------
\title{Hipster-CV}
\author{\LaTeX{} Ninja}
\date{Diciembre 2019}

\pagestyle{empty}
\begin{document}


\thispagestyle{empty}
%-------------------------------------------------------------

\section*{Start}


\header{\bgupper{headerfontbox}{headerfontboxfont}{\bfseries\Huge Daniela Risaro}}{\bg{headerfontbox}{headerfontboxfont}{\large Oceanógrafa}}{\bg{headerfontbox}{headerfontboxfont}{FCEyN-UBA} }{daniufoto.jpg}{headerblue}{2.5cm}{2cm}

%------------------------------------------------

% hier muss die "unsichtbare" Überschrift rein, weil er sonst nicht die Paracols startet... komisch...
\subsection*{}
\vspace{4em}

\setlength{\columnsep}{1.5cm}
\columnratio{0.325}[0.65]
\begin{paracol}{2}
\hbadness5000
%\backgroundcolor{c[1]}[rgb]{1,1,0.8} % cream yellow for column-1 %\backgroundcolor{g}[rgb]{0.8,1,1} % \backgroundcolor{l}[rgb]{0,0,0.7} % dark blue for left margin

\paracolbackgroundoptions

% 0.9,0.9,0.9 -- 0.8,0.8,0.8


\footnotesize
{\setasidefontcolour
\bgupper{cvgreen}{white}{Datos} \\

\\

\bg{cvgreen}{white}{Personales} \\

\begin{tabular}{ll}
\faFemale&Daniela Bel\'en Risaro \\
\faGlobe& Argentina - Italiana   \\
\faBirthdayCake& 1991 \\
\faMapMarker&Buenos Aires \\
\end{tabular}

\bigskip

\bg{cvgreen}{white}{Especialización} \\

Ciencia de datos ~•~ Análisis geoespacial ~•~ Cambio clim\'atico ~•~ Dinámica Oce\'anica

\bigskip
\vspace{.75cm}

\bgupper{cvgreen}{white}{Habilidades} \\

\\


\bg{cvgreen}{white}{Idiomas}
\bigskip


\begin{minipage}[t]{\leftcolwidth}
\begin{tabular}{l | ll}
\textbf{Espa\~nol} &   & {\phantom{x}\footnotesize lengua nativa} \\
\textbf{Ingl\'es} &   & \pictofraction{\faCircle}{cvpurple}{4}{black!30}{1}{\tiny}\\
\textbf{Italiano} &   & \pictofraction{\faCircle}{cvpurple}{1}{black!30}{4}{\tiny}\\
\textbf{Alem\'an} &   & \pictofraction{\faCircle}{cvpurple}{1}{black!30}{4}{\tiny}
\end{tabular}
\end{minipage}

\bigskip

\vspace{.75cm}

\bg{cvgreen}{white}{IT \& programaci\'on} \\

\begin{minipage}[t]{0.3\textwidth}
\begin{tabular}{r @{\hspace{0.5em}}l}
     \bg{skilllabelcolour}{iconcolour}{\LaTeX} & \barrule{0.5}{0.5em}{cvgreen} \\
     \bg{skilllabelcolour}{iconcolour}{python} &  \barrule{0.5}{0.5em}{cvgreen}\\
     \bg{skilllabelcolour}{iconcolour}{MatLab} & \barrule{0.45}{0.5em}{cvpurple} \\
     \bg{skilllabelcolour}{iconcolour}{C++} & \barrule{0.4}{0.5em}{cvpurple} \\
     \bg{skilllabelcolour}{iconcolour}{Git} & \barrule{0.35}{0.5em}{cvpurple} \\
     \bg{skilllabelcolour}{iconcolour}{R} & \barrule{0.35}{0.5em}{cvpurple} \\      \bg{skilllabelcolour}{iconcolour}{HTML} & \barrule{0.3}{0.5em}{cvpurple} \\
     \bg{skilllabelcolour}{iconcolour}{CSS} & \barrule{0.3}{0.5em}{cvpurple} \\
     \end{tabular}

\bigskip
\vspace{.5cm}
\bg{cvgreen}{white}{Blandas}\\

\bubblediagram{{\textbf{Curiosidad}}, comunicaci\'on, orden, aprendizaje\\r\'apido, empat\'ia, planificaci\'on, motivaci\'on, \textbf{Trabajo en}\\ \textbf{equipo}}

% \hspace{3cm} \color{labelcolour}{OS:} \hspace{0.5em}\icon{\faWindows}{labelcolour}{\Large} \hspace{0.5em} \icon{\faLinux}{labelcolour}{\Large} 
\bigskip

\end{minipage}

% \bigskip
% \scalebox{0.7}{
% \iconcross{\Huge}{white}{cvred}{\color{black!30}\faBook}{\href{mailto:dbrisaro@gmail.com}{\faEnvelopeO}}{\faPhone}{\faCode}}


\phantom{turn the page}
}
%-----------------------------------------------------------
\switchcolumn

\small

\section*{Sobre m\'i}
Soy una Ocean\'ografa fisica especializada en la investigación del clima en el mar. Me dedico a recolectar y analizar datos del Mar Argentino para entender cómo y por qué se mueve el océano. Tengo una sólida formación físico-matemática y amplia experiencia en el procesamiento y análisis estadístico de información geoespacial de distintas bases de datos. He participado en actividades de manejo, gestión, divulgación y coordinación de grupos de trabajo. Disfruto el trabajo en grupo y estar en constante aprendizaje y deconstrucción.

\section*{Estudios}

\begin{tabular}{r| p{0.495\textwidth} c}
    \cvevent{2019-presente}{Cs de la computación}{Facultad de Cs Exactas y Naturales - UBA}{\color{white}}{}{}\\    
    \cvevent{2015-2020}{Doctorado en Cs de la Atm\'osfera y los Oc\'eanos}{Facultad de Cs Exactas y Naturales - UBA}{\color{white}}{Titulo de tesis: Las tendencias de largo plazo de la temperatura superficial del mar alrededor de	Sudamérica y su posible impacto ecológico}{} \\    
    \cvevent{2009-2015}{Licenciatura en Oceanograf\'ia}{Facultad de Cs Exactas y Naturales - UBA}{\color{white}}{Promedio: 8.7}{} \\
\end{tabular}

\section*{Gestión}
\begin{tabular}{>{\footnotesize\bfseries}r >{\footnotesize}p{0.595\textwidth}}
    2020 & Youth Ambassador of the Atlantic Ocean Marine conservation UE \\
    2016-2018 & Representante Consejo departamental - FCEyN-UBA - Claustro de Graduados  \\
    2014-2015 & Representante Consejo Departamental - FCEyN-UBA - Claustro de Estudiantes \\
    2012-2014 & Divulgadora Equipo de Popularización - FCEyN-UBA \\

\end{tabular}
    
\section*{Docencia}
Varias posiciones en el Departamento de Ciencias de la Atm\'osfera y los Oc\'eanos. Facultad Cs. Exactas y Naturales.\\

\begin{tabular}{>{\footnotesize\bfseries}r >{\footnotesize}p{0.595\textwidth}}
    2016-2020 & Jefa de Trabajos Pr\'acticos\\
    2015-2016 & Ayudante de primera\\
    2014-2015 & Ayudante de segunda
\end{tabular}

\section*{Investigación}
\begin{tabular}{>{\footnotesize\bfseries}r >{\footnotesize}p{0.57\textwidth}}
    2015-2020 & Becario de posgrado - CONICET\\
    2014-2015 & Becario de grado - Inter-American Institute for Global Change Research (IAI)\\
\end{tabular}
\footnotesize
\renewcommand\labelitemi{\tiny$\bullet$}
    \begin{itemize}
        \item Ánálisis estadísticos de series temporales (regresiones, análisis espectrales y armónicos)
        \item Implementation de métodos estadísticos a datos oceanográficos y atmosféricos distribuídos espacialmente
        \item Participación en cursos y workshops nacionales e internacionales.
        \item Herramientas: Python (Pandas, Matplotlib, Seaborn, Scikit-learn), GIS, Matlab, Git, PostgreSQL, HTML, CSS, Latex, MS Office
    \end{itemize}


%\cventry{2015-present}{National Scientific and Technical
%Research Council (CONICET) Scholarship}{}{}{}{}

%\cventry{2018}{Nippon Foundation-POGO-GEOMAR Fellowship for training on-board RV Meteor cruise M148}{}{}{}{}
\footnotesize
% Investigador de grado ~•~ Inter-American Institute for Global Change Research (IAI) ~•~ Servicio de Hidrografía Naval ~•~2014-2015\\
% Investigador de posgrado ~•~ CONICET ~•~ Servicio de Hidrografía Naval ~•~ 2015-2020\\
% Pasantía profesional ~•~ Buque RV-Meteor ~•~ 2018
%\section*{Talks}
%\begin{tabular}{>{\footnotesize\bfseries}r >{\footnotesize}p{0.6\textwidth}}
%    Nov. 1726 & ``How I lost my ship (\& and how to get it back)'', at: \emph{Annual Pirate's Conference} in Tortuga, Nov. 1726.
%\end{tabular}
%\end{minipage}

\vfill{} % Whitespace before final footer

%----------------------------------------------------------------------------------------
%	FINAL FOOTER
%----------------------------------------------------------------------------------------
\setlength{\parindent}{0pt}
\begin{minipage}[t]{\rightcolwidth}
\begin{center}\fontfamily{\sfdefault}\selectfont \color{black!70}
{\small Daniela Bel\'en Risaro \icon{\faEnvelopeO}{cvpurple}{} dbrisaro@gmail.com \icon{\faMapMarker}{cvpurple}{} CABA, Buenos Aires \newline\icon{\faPhone}{cvpurple}{}+5411 3479-5035  \icon{\faTwitter}{cvpurple}{}\protect\url{dbrisaro}\icon{\faGithub}{cvpurple}{} dbrisaro
}
\end{center}
\end{minipage}


\end{paracol}

\end{document}
